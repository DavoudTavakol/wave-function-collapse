\documentclass[12pt]{report}
% Bibliotheken und Pfade
\usepackage{xcolor} %Farben
\usepackage{graphicx} %Bilder
\usepackage{subfig} %Bilder erweiterung
\usepackage{hyperref} %Links
\usepackage{geometry} %margins
\usepackage[utf8]{inputenc} %Standard!
\usepackage{mathptmx} %Wir benutzen hier eine Mathebibiliothen um Times new Roman zu verwenden
\usepackage[T1]{fontenc} %Standard!
\usepackage{setspace} %zeilenabstände
\usepackage{float}
\usepackage{listings} %code darstellung
\usepackage{blindtext} %Blindtext1
\usepackage{lipsum} %blindtext2
%für deutsch
\usepackage[ngerman]{babel}

\usepackage{titlesec} %zum entfernen der "Kapitel N" header

%wir wollen formatieren{was genau?}[optionaler argument: im block satz.]{kapitel nummerierung und font}{der kapitelname mit angehängtem punkt .}{abstand zwischen nummer und name}{größe titel}
\titleformat{\chapter}[block]{\normalfont\Large\bfseries}{\thechapter.}{1em}{\Large}

\titlespacing*{\chapter}{0pt}{40pt}{30pt} % this alters "before" spacing (the second length argument) to 0
%\titlespacing*{\chapter}{0pt}{50pt}{40pt} % default

\graphicspath{ {images/} }
\geometry{top=3.0cm}


\hypersetup{
    colorlinks,
    linkcolor={black!50!black},
    citecolor={blue!50!black},
    urlcolor={blue!80!black}
}


% Baue die Titel Struktur + Inhalt
\title{
{Bachelorarbeit}\\
{\vspace{10mm}}
{\small Wave-Function-Collapse}\\
{\small Funktionalität und Anwendungsfälle des WFC-Algorithmus}\\
{\small TH-Nürnberg Georg-Simon-Ohm}\\
}
% Setze Author und Datum
\author{Davoud Tavakol}
\date{29.12.2022}

% Start Dokument
\begin{document}

% Erstelle den Titel (Dieser Befehlt setzt dann auch den Author und das Datum!)
\maketitle

{\let\clearpage\relax\chapter*{Abstract}}

zum schluss..

% Automatisches Inhaltsverzeichniss
\tableofcontents

{\let\clearpage\relax\chapter{Abbildungsverzeichnis}}

{\let\clearpage\relax\chapter{Abkürzungsverzeichnis}}

WFC\dotfill{Wave-Function-Collapse}\\
PCG\dotfill{Procedural-Content-Generation}\\
CSP\dotfill{Constraint-Satisfaction-Problem}\\

\chapter{Einleitung}

TODO AM ENDE.
Die automatische Generierung von Inhalten wie Texte, Images oder Modellen ist heutzutage Standard in vielen Bereichen der Industrie.
Um solche Inhalte vordefinierten Parametern zu erstellen werden vor allem zwei Methoden zur Generierung verwendend.
AI's {(Künstliche Intelligenzen)} wie ChatGPT und Algorithmen.
Der logische Vorteil von solchen Tools ist es, das diese in kürzester Zeit qualitative Resultate Generieren können und auch wie oben erwähnt vordefinierte Parameter als Input erhalten können,
um die Ergebnisse für ihren gebrauch anzupassen.
In dieser Bachelorarbeit werde ich mich auf den Wave-Function-Collapse Algorithmus, dessen Funktionsweise und Anwendungsfälle fokussieren.


\chapter{Hintergrund}

\section{Textursynthesen im Vergleich}

Es gibt viele Möglichkeiten Textursynthese mit Algorithmen zu erzielen.
Die meisten dieser Methoden basieren auf demselben Grundprinzip aus kleineren Input-Images größere oder gleich große Output-Images zu generieren.
Wichtig hierbei ist, dass das Muster des Output-Images lokal ähnlich oder gleich ist.
Das wird größtenteils dadurch erzielt das aus dem Input-Image kleinere Subimages extrahiert werden {(z.B. 5 x 5 Pixel Fenster)}.
Bei den verfahren, wo die lokale Ähnlichkeit nicht 1-zu-1 bzw. pixelgenau stattfindet, werden die Pixel und deren Farbwert oft nach Grundlage der Abstandsmetrik {(z.B. dem euklidischen Abstand von Pixelfarbvektoren)} beurteilt.
Solche Verfahren finden meistens in der rein visuellen Computergrafik Anwendung.
Diese Methodiken haben große Nachteile im Gegensatz zu Algorithmen wo das lokale Muster des Outputs pixelgenau dem Input-Image gleicht.
Gerade bei PCG {(Procedural-Content-Generation)} kann die Pixelgenauigkeit von großen Nutzen sein da dadurch Abgrenzungen der Pixel innerhalb des Output-Images klar definiert sein können.
{[1]}
Der WFC von Gumin ist lose an der Quantenmechanik angelehnt.
Das liegt daran, dass bei der Synthese von WFC in jeder Zelle des $N\times N$ Output-Images theoretisch jedes Muster / Pixelwert vorkommen kann bevor sie final festgelegt werden.
Dieser Zustand nennt sich \textit{Superposition}.
Jede Zelle hat mehrere Eigenwerte \textit{(eigenstates)} und somit auch eine maximale Entropie bzw. einen maximalen Informationsgehalt.
Sobald eine Zelle bekannt wird \textit{(Observation)} und damit nur einen Eigenwert besitzt, dann wird die Entropie aller anderen Zellen angepasst.
{(Auf dieses Verhalten wird später tiefer eingegangen)}. {[2]}
Gumin hat sich von der Arbeit von Paul Merell Inspirieren lassen, obwohl dieser sich Hauptsächlich mit der Generierung von 3D-Modellen befasst hatte.
Bei seinem Verfahren werden die Modelle mithilfe von bereits erstellten Bausteinen zusammengesetzt.
Das ist dahingehen wichtig da in vielen Textursynthesen gerade bei den Übergängen die Pixel sich Mischen und somit sich Artefakte bilden.
Dieser Verhalten ist bei WFC und dem Verfahren von Paul Merell nicht möglich da es sich um eine diskrete Synthese handelt.
Jedes lokale Muster ist immer im Input wiederzufinden. {[1]}{[3]}{[4]}

\section{Constraint-Satisfaction-Problem}

Was ist ein Constraint-Satisfaction-Problem? {(CSP)}
Grundsätzlich beschreiben CSP's Gruppen von Objekten denen Variablen zugeteilt sind.
Diesen Variablen sind Regeln, sogenannte \textit{(constraints)}, auferlegt die erfüllt werden müssen.
Jeder dieser Variablen hat zu Beginn eine Superposition und kann damit jeden wert enthalten.
Die Aufgabe von Algorithmen zum Lösen von CSP's \textit{(solver)} ist es einen Zustand \textit{(State)} zu finden in denen alle constraints erfüllt sind und jeder Variable nur noch ein wert zugeordnet ist.
Für solche Probleme finden sich oft bei der Künstlichen Intelligenz und aus dem Operations Research. {[5]}
Im Fall von WFC sind die Objekte, denen die Variablen zugeteilt sind, die einzelnen Bereiche im Output-Image.
Jeder diese Bereiche muss ein lokales Muster aus dem Input zugeordnet werden.
Immer, wenn einem Bereich ein wert zugeordnet wird, dann werden auch die benachbarten Bereiche damit beeinflusst.
Der Prozess, wenn sich eine Gruppe aus Superpositionen mit mehreren Eigenwerten zu einem einzelnen Eigenwert aufgrund von Interaktion mit der Außenwelt \textit{(Observation)} reduziert,
nennt sich Wave-Function-Collapse. {[6]}
Während dem Prozess einen gültigen State für das CSP zu finden,dann gibt es immer Situationen in dem es mehrere gültige Optionen für eine Variable gibt.
Wenn so eine Situation auftritt, dann haben verschiedene Solver verschiedene Ansätze.
Einige Algorithmen wähle zufällig eine der möglichen Werte von momentan zulässigen Optionen.
Bei so einem Ansatz kann es sein das der Algorithmus nicht auf einen Zustand kommen kann, in dem alle constraints erfüllt werden können.
In so einem Fall gibt es Rücksetzverfahren \textit{(Backtracking)} bei dem der Algortihmus zu seinem letzten Ergebniss zurückfällt und ein anderen Wert für die Variable setzt um aus dem
ungültigen Zustand zu kommen.
Andere Algorithmen verwenden sogenannte Heuristiken um die möglichkeit eines ungültigen Zustandes zu reduzieren und das verwenden eines Backtrackings zu beschleunigen. {[1]} 




{\let\clearpage\relax\chapter{Begriffserklärung}}

{\let\clearpage\relax\chapter{Theorie}}
{\let\clearpage\relax\chapter{Stand der Forschung}}
{\let\clearpage\relax\chapter{Ergebnisse}}
{\let\clearpage\relax\chapter{Diskussion der Ergebnisse}}

{\let\clearpage\relax\chapter{Fazit}}

{\let\clearpage\relax\chapter{Literaturverzeichnis}}
{[1]} \url{https://adamsmith.as/papers/wfc_is_constraint_solving_in_the_wild.pdf}\\
{[2]} \url{https://en.wikipedia.org/wiki/Wave_function_collapse}\\
{[3]} \url{https://paulmerrell.org/wp-content/uploads/2021/06/thesis.pdf}\\
{[4]} \url{https://github.com/mxgmn/WaveFunctionCollapse}\\
{[5]} \url{https://en.wikipedia.org/wiki/Constraint_satisfaction_problem}\\
{[6]} \url{https://en.wikipedia.org/wiki/Wave_function_collapse}\\
\url{https://www.cv-foundation.org/openaccess/content_cvpr_2016/papers/Gatys_Image_Style_Transfer_CVPR_2016_paper.pdf}\\
\url{https://www.th-nuernberg.de/fileadmin/global/Gelenkte_Doks/Fak/SW/SW_0600_HR_Leitfaden_WA_public.pdf}\\
\url{https://www.ghost-writing.net/wissenschaftliche-arbeit-auf-englisch-verfassen/}\\
\url{https://www.youtube.com/watch?v=rI_y2GAlQFM&t=1135s&ab_channel=TheCodingTrain}\\
\url{https://users.informatik.haw-hamburg.de/~abo781/abschlussarbeiten/ba_westfalen.pdf}\\
\url{https://users.informatik.haw-hamburg.de/~abo781/abschlussarbeiten/ba_dzaebel.pdf}\\
\url{http://people.csail.mit.edu/celiu/Patch-based%20Texture%20Synthesis/Index.htm}\\
\url{https://www2.eecs.berkeley.edu/Research/Projects/CS/vision/papers/efros-iccv99.pdf}\\


{\let\clearpage\relax\chapter{Anhang}}
{\let\clearpage\relax\chapter{Eidesstattliche Erklärung}}
\end{document}